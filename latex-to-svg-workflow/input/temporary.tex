\documentclass{standalone}
\usepackage{ProfLycee}
\useproflyclib{ecritures}
\usepackage{tikz}
\begin{document}

\begin{tikzpicture}[x=0.5cm,y=0.5cm, %unités
  xmin=0,xmax=12,xgrille=1,xgrilles=1, %axe Ox
  ymin=0,ymax=12,ygrille=1,ygrilles=1] %axe Oy
  \FenetreSimpleTikz%
  <Police=\small>{auto}%
  <Police=\small>{auto} %repère

  \foreach \x in {0,...,10} {
    \draw[fill=CouleurVertForet] (\x, \x+1) circle (2pt);
  }
  \foreach \x in {0,5,10} {
    \draw[fill=red] (\x, 0.2*\x+4) circle (2pt);
  }
  \foreach \x in {0,2,4,6,8,10} {
    \draw[fill=blue] (\x, -0.5*\x+7) circle (2pt);
  }
\end{tikzpicture}

\end{document}



Dans le repère orthonormé ci-dessous, on a représenté quelques termes de trois suites arithmétiques.

Pour chacune d'elle, déterminer le premier terme, la raison ainsi que l'expression de $u_n$ en fonction de $n$.
Donner la valeur de $u_3$ et $u_6$.
\begin{center}

  \begin{tikzpicture}[x=0.5cm,y=0.5cm, %unités
    xmin=0,xmax=12,xgrille=1,xgrilles=1, %axe Ox
    ymin=0,ymax=12,ygrille=1,ygrilles=1] %axe Oy
    \FenetreSimpleTikz%
    <Police=\small>{auto}%
    <Police=\small>{auto} %repère

    \foreach \x in {0,...,10} {
      \draw[fill=CouleurVertForet] (\x, \x+1) circle (2pt);
    }
    \foreach \x in {0,5,10} {
      \draw[fill=red] (\x, 0.2*\x+4) circle (2pt);
    }
    \foreach \x in {0,2,4,6,8,10} {
      \draw[fill=blue] (\x, -0.5*\x+7) circle (2pt);
    }
  \end{tikzpicture}
\end{center}
