\documentclass{article}
\usepackage{ProfLycee}
\usepackage{tikz}
\begin{document}

\begin{enumerate}
    \item Elle modélise le nombre de véhicules présents au péage en fonction de l'heure de la matinée $t$, par la fonction définie sur $[0 ; 12]$ par :
    $$f(t) = -0,3t^3 + 5,4t^2 - 18t + 19$$
    \begin{enumerate}
      \item Calculer la fonction dérivée $f'(t)$ puis dresser le tableau de variation de la fonction $f$ sur $[0 ; 12]$.
      \item En déduire l'heure de l'affluence maximale de la matinée. Quel est alors le nombre de véhicules présents au péage à cet instant?
    \end{enumerate}
  
    \item Pour l'affluence de fin de journée, le modèle choisi est la fonction $g$ définie sur $[14 ; 23]$ par :
    $$g(t) = -2t^2 + 74t - 630$$
    \begin{enumerate}
      \item Calculer la fonction dérivée $g'(t)$ puis dresser le tableau de variation de la fonction $g$ sur $[14 ; 23]$.
      \item Le responsable du péage sait que lorsque l'affluence dépasse 50 véhicules, il lui est nécessaire, pour fluidifier le trafic, d'ouvrir toutes les voies de paiement.
      
      Déterminer la tranche horaire durant laquelle toutes les voies doivent être ouvertes.
    \end{enumerate}
  \end{enumerate}


  

\end{document}